\documentclass[10pt, a4paper]{article}

% Text packages
\usepackage[french]{babel}
\usepackage[utf8]{inputenc}
\usepackage{lmodern} % Latin founts

% Fonts packages
\usepackage{ifxetex}
\ifxetex
    \usepackage{fontspec}
    \setmainfont{OpenDyslexic}
\else
    \usepackage[T1]{fontenc}
\fi

% Math fonts
\usepackage{amsmath}
\usepackage{amsfonts}
\usepackage{latexsym}

% Theorem fonts
\usepackage{amsthm}

% Title import
\usepackage{authoraftertitle}

% Clickable link
\usepackage{hyperref}
\hypersetup{
    colorlinks=true,
    citecolor=blue,
    filecolor=black,
    linkcolor=black,
    urlcolor=blue
}

% Image side by side
\usepackage{subcaption}

% Text color
\usepackage{xcolor}

\usepackage{xspace}

% Image
\usepackage{graphicx}
\graphicspath{ {./images/} }
\usepackage{bookmark}

% Code import
\usepackage{minted}
\usemintedstyle{emacs} % borland

\title{PaP}
\author{BERASATEGUY Tanguy, GOEDEFROIT Charles}

\begin{document}

\begin{titlepage}
	\centering
    \ {} % important
	\vfill
	\vspace{1cm}
	{\scshape\LARGE\MyTitle\par}
	\vspace{0.5cm}
	{\huge\bfseries Projet : rapport2\par}
	\vspace{0.5cm}
	{\Large 4TIN804U\par}
	\vspace{1cm}
	\MyAuthor
	\vfill
	{\large2021-2022\par}
\end{titlepage}

\newpage

\tableofcontents

\newpage

\section{ILP optimization (4.1)}

On a fait les modifications :

Pour \emph{ssandPile\_do\_tile\_opt()} on a retiré les appels à \em{table(out, i, j)}
pour passer par une variable intermediaire result. Cette modification permet au compilateur
de vectoriser car il peut maintenant facilement voir que les différentes lignes peuvent être
calculées en parallele.\\

Nous avons modifié ces lignes :
\inputminted[
    frame=lines,
    framesep=2mm,
    baselinestretch=1.2,
    fontsize=\footnotesize,
    linenos,
    firstline=1,
    lastline=25
]{diff}{codes/sync_do_tile_opt.c}

Le code de la fonction final :
\inputminted[
    frame=lines,
    framesep=2mm,
    baselinestretch=1.2,
    fontsize=\footnotesize,
    linenos,
    firstline=27,
    lastline=45
]{c}{codes/sync_do_tile_opt.c}

Nous avons verifié et on obtient le même résultat et le même nombre d'iterations (69190)
avec la version par défaut et la version opt.\\
\begin{figure}[H]
    \centering
    \begin{subfigure}{.4\textwidth}
        \includegraphics[width=1\linewidth]{dump-ssandPile-seq-dim-512-iter-69190_default}
        \caption{\small{avec do\_tile\_default}}
        % \label{fig:ssandPile_default}
    \end{subfigure}
    \begin{subfigure}{.4\textwidth}
        \includegraphics[width=1\linewidth]{dump-ssandPile-seq-dim-512-iter-69190_opt}
        \caption{\small{avec do\_tile\_opt}}
        % \label{fig:ssandPile_opt}
    \end{subfigure}
    \caption{Verification du résultats pour ssandPile}
\end{figure}

Le gain de performance est de $2.37$ car $\frac{178812}{75408}$
\\[0.5cm]

Pour \emph{asandPile\_do\_tile\_opt()} on a retiré les appels à \em{atable(i, j)} pour passer par une
variable intermédiaire result. Cette modification permet au compilateur de vectoriser car il peut
maintenant facilement voir que les différentes lignes peuvent être calculés en parallèle.

Nous avons modifié ces lignes :
\inputminted[
    frame=lines,
    framesep=2mm,
    baselinestretch=1.2,
    fontsize=\footnotesize,
    linenos,
    firstline=1,
    lastline=24
]{diff}{codes/async_do_tile_opt.c}

Le code de la fonction final :
\inputminted[
    frame=lines,
    framesep=2mm,
    baselinestretch=1.2,
    fontsize=\footnotesize,
    linenos,
    firstline=26,
    lastline=46
]{c}{codes/async_do_tile_opt.c}

Nous avons verifié et on obtient le même résultat et le même nombre d'iterations (34938)
avec la version par défaut et la version opt.\\
\begin{figure}[H]
    \centering
    \begin{subfigure}{.4\textwidth}
        \includegraphics[width=1\linewidth]{dump-asandPile-seq-dim-512-iter-34938_default}
        \caption{\small{avec do\_tile\_default}}
        % \label{fig:asandPile_default}
    \end{subfigure}
    \begin{subfigure}{.4\textwidth}
        \includegraphics[width=1\linewidth]{dump-asandPile-seq-dim-512-iter-34938_opt}
        \caption{\small{avec do\_tile\_opt}}
        % \label{fig:asandPile_opt}
    \end{subfigure}
    \caption{Verification du résultats pour asandPile}
\end{figure}

Le gain de performance est de $1.2$ car $\frac{37990}{31405}$

\section{OpenMP implementation of the synchronous version (4.2)}

Pour \emph{ssandPile\_compute\_omp()} on a fait en sorte de parcourir chaque grain de sable
puis on a ajouté un \em{pragam omp for} pour que chaque grain de sable soit calculé par un thread.\\

Le code de la fonction :
\inputminted[
    frame=lines,
    framesep=2mm,
    baselinestretch=1.2,
    fontsize=\footnotesize,
    linenos
]{c}{codes/sync_omp.c}

Pour \emph{ssandPile\_compute\_omp\_tiled()} on a dupliqué la version tile pour y ajouter
un \em{pragam omp for} avec un \em{collapse(2)} pour que les tuiles soient calculés par des threads.\\

La heatmap montre que en moyenne, le schedule static est meilleur que le dynamique,
cependant on cherche à résoudre un problème précis, donc on va chercher l'accélération maximale.
Ici ce sera avec des tuiles de width 512 et de height 4 en schedule dynamique.
\begin{figure}[H]
    \centering
    \includegraphics[width=1\linewidth]{ssand_omp_tiled}
\end{figure}

Pour \emph{ssandPile\_compute\_omp\_taskloop()} on a dupliqué la version tile pour y ajouter
un \em{pragam omp single}, pour qu'un thread definisse les taches, et un \em{pragam omp task}
devant l'appelle de la fonction \em{do\_tile()} pour que le calcule des tuiles se trouve dans un thread.\\

On voit que \em{taskloop} avec des tuile de taille 64 et 32 est presque aussi bonne que \em{omp\_tile} avec
des tuiles de taille 64, 32 et 16. Et que \em{taskloop} n'est pas très efficace avec des tuiles de taille 8 ou 16.
\begin{figure}[H]
    \centering
    \includegraphics[width=1\linewidth]{ssand-xp-all_taskloop.pdf}
    \caption{\small{ssandPile comparaison de taskloop avec toutes les variants}}
    % \label{fig:ssand-xp-all}
\end{figure}

\section{OpenMP implementation of the asynchronous version (4.3)}

Pour paralléliser avec \emph{asandPile\_compute\_omp\_tiled}, il a fallu créer 4 nids de boucles afin d'éviter
les lectures/écritures concurrentes.
On a donc un nid de boucle pour les lignes et colonnes impaires,
un pour les lignes impaires et colonnes paires,
un pour les lignes paires et colonnes impaires,
et un pour les lignes et colonnes paires.

Le code de la fonction final :
\inputminted[
    frame=lines,
    framesep=2mm,
    baselinestretch=1.2,
    fontsize=\footnotesize,
    linenos,
    firstline=1,
    lastline=23
]{c}{codes/async_omp_tiled.c}
\emph{Le nid BLEU est répété 4 fois avec des initialisations de x et y qui diffèrent}

\begin{figure}[H]
    \centering
    \begin{subfigure}{.4\textwidth}
        \includegraphics[width=1\linewidth]{dump-asandPile-seq-dim-512-iter-34938_default}
        \caption{\small{version seq}}
        % \label{fig:asandPile_default}
    \end{subfigure}
    \begin{subfigure}{.4\textwidth}
        \includegraphics[width=1\linewidth]{dump-asandPile-omp_tiled-dim-512-iter-34954}
        \caption{\small{version omp\_tiled}}
        % \label{fig:asandPile_opt}
    \end{subfigure}
    \caption{Verification du résultats pour asandPile}
\end{figure}

\begin{figure}[H]
    \centering
    \includegraphics[width=1\linewidth]{Heatmap-asand_omp}
    \caption{\small{heatmap comparaison omp\_tiled \& tiled}}
    % \label{fig:heatmaplazy}
\end{figure}

Les expériences montrent que pour certaines tailles de tuiles, il n'y a pas de résultat d'accélération,
et après vérification, le programme ne fonctionne pas sur ces tailles là.
Néanmoins, avec ces résultats, ont en déduit que le schedule static,1 avec une width de 32 et une height
de 4 est la meilleure option.

\begin{figure}[H]
    \centering
    \includegraphics[width=1\linewidth]{asand-xp-all_8_16.pdf}
    \includegraphics[width=1\linewidth]{asand-xp-all_32_64.pdf}
    \caption{\small{asandPile comparaison de toutes les variants}}
    % \label{fig:asand-xp-all}
\end{figure}

D'après ces schémas, l'accélération maximale est toujours donnée pour 10 threads peu importe la taille de
tuile donnée (8*8, 16*16, 32*32, 64*64).
On voit également une chute de performances vers 25 threads, c'est parcequ'on était sur \emph{kira}
qui est une machine 24 coeurs.


\section{Lazy OpenMP implementations (4.4)}

Pour la version asynchrone, nous avons repris la version précédente et ajouté deux tableaux
servant successivement de lecture et d'écriture pour tester si les tuiles autour ont été modifiés à
l'itération précédente.

\begin{figure}[H]
    \centering
    \begin{subfigure}{.4\textwidth}
        \includegraphics[width=1\linewidth]{dump-asandPile-seq-dim-512-iter-34938_default}
        \caption{\small{avec do\_tile\_default}}
        % \label{fig:asandPile_default}
    \end{subfigure}
    \begin{subfigure}{.4\textwidth}
        \includegraphics[width=1\linewidth]{dump-asandPile-omp_lazy-dim-512-iter-34954}
        \caption{\small{version omp\_lazy}}
        % \label{fig:asandPile_opt}
    \end{subfigure}
    \caption{Verification du résultats pour asandPile}
\end{figure}

\begin{figure}[H]
    \centering
    \includegraphics[width=1\linewidth]{Heatmap-asand_omp_lazy}
    \caption{\small{heatmap comparaison omp\_lazy \& lazy}}
    % \label{fig:heatmaplazy}
\end{figure}

Les expériences montrent que pour certaines tailles de tuiles, il n'y a pas de résultat d'accélération,
et après vérification, le programme ne fonctionne pas sur ces tailles là.
Néanmoins, avec ces résultats, ont en déduit que le schedule static,1 avec une width de 32 et
une height de 4 est la meilleure option.

\section{Résultats produit a la fin}

\begin{figure}[H]
    \centering
    \includegraphics[width=1\linewidth]{ssand-xp-all}
    \caption{\small{ssand all variants experiments}}
\end{figure}

\begin{figure}[H]
    \centering
    \includegraphics[width=1\linewidth]{ssand-xp-TiledVsLazy}
    \caption{\small{ssand tiled vs lazy experiments}}
\end{figure}

\begin{figure}[H]
    \centering
    \includegraphics[width=1\linewidth]{ssand-xp-OmpTiledVsOmpLazy}
    \caption{\small{ssand omp\_tiled vs omp\_lazy experiments}}
\end{figure}

\begin{figure}[H]
    \centering
    \includegraphics[width=1\linewidth]{asand-xp-all}
    \caption{\small{asand all variants experiments}}
\end{figure}

\begin{figure}[H]
    \centering
    \includegraphics[width=1\linewidth]{asand-xp-TiledVsLazy}
    \caption{\small{asand tiled vs lazy experiments}}
\end{figure}

\begin{figure}[H]
    \centering
    \includegraphics[width=1\linewidth]{asand-xp-OmpTiledVsOmpLazy}
    \caption{\small{asand omp\_tiled vs omp\_lazy experiments}}
\end{figure}

\end{document}